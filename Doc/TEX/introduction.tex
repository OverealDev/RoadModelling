
	A road interchange is a road junction that allows the interchange of traffic between multiple roadways. Roads interchanges use grades separation, which permits the vehicles to move through the junction fluidly, without interrupting the flow of the traffic stream, which may reduces the number of road accidents.
Grade-separated road junctions generally need a lots of space, are complicated and costly. That’s why optimization of the used space is financially interesting. Also, road interchanges are build contextually in already existing areas where space is limited. Thus, civil engineers have to find efficient ways to manage complex interchanges of traffic lines while respecting the unique spatial constraints.
\newline
Apart from the purpose of interconnecting roadways, road interchanges are also playing a role of speed regulation by imposing curves : the tighter the curvature, the slower the speed. Moreover, due to the eventual difference of height of surface transport axes, a road interchange can allows animal to cross roadways and highways more safely, which might diminish the number of accidents.
This way, stakes of road interchanges are numerous, such that patents on road interchanges had been filled. In particular, an intersection called “PINAVIA” have been created by an engineer called “Stanislaus Buteliauskas” in 2004 and patented in 2008 [1]. This road junction is designed for high-intensity roads, for the main purpose of reducing traffic congestion, increasing junction capacities, and indirectly reducing environmental pollution [1]. This new class of road junction is patented as an innovation, but it’s difficult to figure out if such a type of road interchange has been built elsewhere in the world. The inventor must be sure that such an innovative interchanges is not already built somewhere else. Reciprocally, civil engineers have to be careful when it comes to design without permission a road interchange that is already patented. One way to figure it out is to manually review the entire world road map and check every single road interchanges, which is ridiculously fastidious. Another way to do so is to create a algorithm able to automatically scan road maps to be able to detect grade-separated road junctions, in order to classify them. To do so, I thought necessary to be able to model roads based on a road map dataset, which is basically spatially ordered points, sample of the real road. Creating such a model is one of the main goal of this coursework. The code can be found at on gitHub : \url{https://github.com/OverealDev}.
